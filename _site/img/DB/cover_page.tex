\documentclass[11pt]{article}
\usepackage[a4paper,margin=2.5cm]{geometry}
\usepackage{fontspec}
\usepackage{xeCJK}
\usepackage{graphicx}
\usepackage{tikz}
\usepackage{fancyhdr}

% 폰트 설정
\setmainfont{DejaVu Serif}
\setCJKmainfont{Apple SD Gothic Neo}

% 헤더/푸터 제거
\pagestyle{empty}

\begin{document}

% 표지 페이지
\begin{titlepage}
\begin{tikzpicture}[remember picture,overlay]

% 편집 날짜 (오른쪽 상단)
\node[anchor=north east] at ([xshift=-1cm,yshift=-1cm]current page.north east) {
    \footnotesize 25.07.29 편집
};

% 로고 영역 (오른쪽 상단)
\node[anchor=north east] at ([xshift=-1cm,yshift=-3cm]current page.north east) {
    \begin{minipage}{8cm}
        \centering
        % CMS 로고 자리 (실제 로고로 교체 필요)
        \fbox{\makebox[3cm][c]{\parbox{3cm}{\centering CMS\\LOGO}}}
        \hspace{0.5cm}
        % 서울대 로고 자리 (실제 로고로 교체 필요)  
        \fbox{\makebox[3cm][c]{\parbox{3cm}{\centering SNU\\LOGO}}}
    \end{minipage}
};

% 제목 (중앙)
\node[anchor=center] at (current page.center) {
    \begin{minipage}{15cm}
        \centering
        \vspace{2cm}
        
        {\Huge\bfseries LRSM WR→tb Channel Analysis}
        
        \vspace{1.5cm}
        
        {\Large Decay channel: WR→e/μ + N→e/μ + WR*→tb}
        
        \vspace{3cm}
        
        {\large 
        Left-Right Symmetric Model을 이용한\\
        WR 입자의 top-bottom 채널 분석 연구
        }
        
        \vspace{2cm}
        
        {\large 
        서울대학교 물리천문학부\\
        입자물리학 연구실
        }
        
    \end{minipage}
};

% 하단 정보
\node[anchor=south] at ([yshift=3cm]current page.south) {
    \begin{minipage}{15cm}
        \centering
        {\large 2025년 7월}
    \end{minipage}
};

\end{tikzpicture}
\end{titlepage}

\end{document}